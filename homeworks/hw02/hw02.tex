\documentclass[12pt]{article}

\include{preamble}

\newtoggle{professormode}
\toggletrue{professormode} %STUDENTS: DELETE or COMMENT this line



\title{MATH 341 / 650.3 Spring 2019 Homework \#2}

\author{Professor Adam Kapelner} %STUDENTS: write your name here

\iftoggle{professormode}{
\date{Due in KY604, Tuesday 5PM, February 26, 2019 \\ \vspace{0.5cm} \small (this document last updated \today ~at \currenttime)}
}

\renewcommand{\abstractname}{Instructions and Philosophy}

\begin{document}
\maketitle

\iftoggle{professormode}{
\begin{abstract}
The path to success in this class is to do many problems. Unlike other courses, exclusively doing reading(s) will not help. Coming to lecture is akin to watching workout videos; thinking about and solving problems on your own is the actual ``working out.''  Feel free to \qu{work out} with others; \textbf{I want you to work on this in groups.}

Reading is still \textit{required}. For this homework set, read about the beta-binomial model and conjugacy in Bolstad and read Ch4--7 of McGrayne.

The problems below are color coded: \ingreen{green} problems are considered \textit{easy} and marked \qu{[easy]}; \inorange{yellow} problems are considered \textit{intermediate} and marked \qu{[harder]}, \inred{red} problems are considered \textit{difficult} and marked \qu{[difficult]} and \inpurple{purple} problems are extra credit. The \textit{easy} problems are intended to be ``giveaways'' if you went to class. Do as much as you can of the others; I expect you to at least attempt the \textit{difficult} problems. 

Problems marked \qu{[MA]} are for the masters students only (those enrolled in the 650.3 course). For those in 341, doing these questions will count as extra credit.

This homework is worth 100 points but the point distribution will not be determined until after the due date. See syllabus for the policy on late homework.

Up to 10 points are given as a bonus if the homework is typed using \LaTeX. Links to instaling \LaTeX~and program for compiling \LaTeX~is found on the syllabus. You are encouraged to use \url{overleaf.com}. If you are handing in homework this way, read the comments in the code; there are two lines to comment out and you should replace my name with yours and write your section. The easiest way to use overleaf is to copy the raw text from hwxx.tex and preamble.tex into two new overleaf tex files with the same name. If you are asked to make drawings, you can take a picture of your handwritten drawing and insert them as figures or leave space using the \qu{$\backslash$vspace} command and draw them in after printing or attach them stapled.

The document is available with spaces for you to write your answers. If not using \LaTeX, print this document and write in your answers. I do not accept homeworks which are \textit{not} on this printout. Keep this first page printed for your records.

\end{abstract}

\thispagestyle{empty}
\vspace{1cm}
NAME: \line(1,0){380}
\clearpage
}

\problem{These are questions about McGrayne's book, chapters 4-7.}

\begin{enumerate}

\easysubproblem{Describe four things Bayesian modeling was applied to during WWII and identify the people who developed each application.}\spc{8}

\intermediatesubproblem{What do you think was the main reason Bayesian Statistics fell out of favor at the end of WWII?}\spc{2}


\intermediatesubproblem{Why weren't the leaders of Statistics world in the 1950's able to answer the think-tank's question about the $\prob{\text{war in the next 5 years}}$?}\spc{2}

\easysubproblem{Who was responsible for reviving the interest in Bayesian Statistics post-WWII and why?}\spc{2}


\hardsubproblem{In 1955, there were no midair collisions of two planes. How was the actuary able to estimate that the number would be above zero?}\spc{2}

\easysubproblem{The main attack on Bayesian Statistics has always been subjectivity. Answer the following question how Savage would have answered it: \qu{If prior opinions can differ from one researcher to the next, what happens to scientific objectivity in data analysis?} Do you believe Savage's idea is the way science works in the real world?}\spc{3}


\hardsubproblem{[MA] On page 104, Sharon writes, \qu{Bayesians would also be able to concentrate on what happened, not on what \textit{could} have happened according to Neyman Pearson's samping plan}. (Note that the \qu{Neyman Pearson's samping plan} is synonymous with Frequentist Statistics). Explain (1) how Bayesians concentrate on \qu{what happened} and (2) how Frequentists concentrate on what \qu{\textit{could} have happened} in the context on page 104.}\spc{6}


\easysubproblem{Who were the two tireless champions of Bayesian Statistics throughout the 50's, 60's and 70's and where geographically were they located during the majority of their career?}\spc{2}

\end{enumerate}



\problem{We examine here paternity testing (i.e. answering the question \qu{is this guy the father of my child?}) via the simplistic test using blood types. These days, more advanced genetic methods exist so these calculations aren't made in practice, but they are a nice exercise. 

First a crash course on basic genetics. In general, everyone has two alleles (your genotype) with one coming from your mother and one coming from your father. The mother passes on each of the alleles with 50\% probability and the father passes on each allele with 50\% probability. One allele gets expressed (your phenotype). So one of the genes shone through (the dominant one) and one was masked (the recessive one). Dominant blood types are A and B and the recessive type is o (lowercase letter). The only way to express phenotype o is to have genotype oo i.e. both genes are o. There is an exception; A and B are codominant meaning that blood type AB tests positive for both A and B.

In this case consider a child of blood type B (and negative for A) and the mother of blood type A (and negative for B). Using this \href{http://www.cccoe.net/genetics/blood2.html}{hereditary guide}, we know that the mother's type must be Ao so she passed on an o to the child thus the child got the B from the father. Thus the father had type AB, BB or Bo. I got the following frequency data from \href{http://www.sciencedirect.com/science/article/pii/S1110863011000796}{this paper} (so let's assume this case is in Nigeria in 1998). You can assume $n$ is large enough so that these frequencies are as good as the true probabilities.

\begin{table}[h]
\centering
\begin{tabular}{cc}
Genotype & Frequency \\ \hline
OO	&0.5200 \\
AA	&0.0196 \\
AO	&0.2016 \\
BB	&0.0196 \\
BO	&0.2016 \\
AB	&0.0392 \\
\end{tabular}
\end{table}
} 

\begin{enumerate}


\intermediatesubproblem{What is the probability a stranger passes on a B to the child?}\spc{3}

\hardsubproblem{Bob is the alleged father and he has blood type B but his genotype is unknown. What is the probability he passes on a B to the child? Hint: it is the same as (a) except now}\spc{3}


%\easysubproblem{Assume our prior is 50-50 Bob is the father, the customary compromise between a possibly bitter mother and father. What is the prior odds of Bob being the father? Don't think too hard about this one; it is marked easy for a reason.}\spc{6}

\hardsubproblem{We are interested in the posterior question. What is the probability Bob is the father given the child with blood type B? You can assume a prior of indifference.}\spc{5}

%\hardsubproblem{What is the Bayes Factor here? See (a) and (b).}\spc{5}

\easysubproblem{What is the probability Bob is not the father given the child with blood type B?}\spc{3}

\end{enumerate}


\problem{We will now be looking at the beta-prior, binomial-likelihood Bayesian model and introduce credible regions as well.}

\begin{enumerate}

\easysubproblem{Using the principle of indifference, what should the prior on $\theta$ (the parameter for the Bernoulli model) be?}\spc{1}

\easysubproblem{Let's say $n=6$ and your data is $0,1,0,1,0,1$. What is the likelihood of this event?}\spc{2}

\easysubproblem{Does it matter the order as to which the data came in? Yes/no.}\spc{0.5}

\intermediatesubproblem{Show that the unconditional joint probability (the denominator in Bayes rule) is a beta function and specify its two arguments.}\spc{5}

\intermediatesubproblem{Put your answer from (a), (b) and (d) together to find the posterior probability of $\theta$ given this dataset. Show that it is equal to a beta distribution and specify its parameters.}\spc{5}

\easysubproblem{Now imagine you are not indifferent and you have some idea about what $\theta$ could be a priori and that subjective feeling can be specified as a beta distribution. (1) Draw the basic shapes that the beta distribution can take on, (2) give an example of $\alpha$ and $\beta$ values that would produce these shapes and (3) write a sentence about what each one means for your prior belief. These shapes are in the notes.}\spc{12}

\intermediatesubproblem{Imagine $n$ data points of which you don't know the realization values. Using your prior of $\theta \sim \stdbetanot$, show that $\theta~|~X \sim \betanot{\alpha + x}{\beta + (n - x)}$. Note that $x := \sum_{i = 1}^n x_i$ which is the total number of successes and thereby $n-x$ is the total number of failures.}\spc{10}

\easysubproblem{What does it mean that the beta distribution is the \qu{conjugate prior} for the binomial likelihood?}\spc{3}

\hardsubproblem{Show that if $Y \sim \stdbetanot$ then $\var{Y} = \frac{\alpha\beta}{\squared{\alpha + \beta}(\alpha + \beta + 1)}$.}\spc{11}

\extracreditsubproblem{Prove that $B(\alpha, \beta) = \frac{\Gammaf{\alpha} \Gammaf{\beta}}{\Gammaf{\alpha + \beta}}$.}


\intermediatesubproblem{The posterior is $\theta~|~X \sim \betanot{\alpha + x}{\beta + (n - x)}$. Some say the values of $\alpha$ and $\beta$ can be interpreted as follows: $\alpha$ is considered the prior number of successes and $\beta$ is considered the prior number of failures. Why is this a good interpretation? Writing out the PDF of $\theta~|~X$ should help you see it.}\spc{5}

\intermediatesubproblem{If you employ the principle of indifference, how many successes and failures is that equivalent to seeing a priori?}\spc{3}

\easysubproblem{Why are large values of $\alpha$ and/or $\beta$ considered to compose a \qu{strong} prior?}\spc{2}

\intermediatesubproblem{[MA] What is the weakest prior you can think of and why?}\spc{5}


\hardsubproblem{I think a priori that $\theta$ should be expected to be 0.8 with a standard error of 0.02. Solve for the values of $\alpha$ and $\beta$ based on my a priori specification. }\spc{9}

%\hardsubproblem{Prove that the posterior predictive distribution is $X^*~|~X \sim \bernoulli{\frac{x + \alpha}{n + \alpha + \beta}}$. MA students --- do this yourself. Other students --- use my notes and justify each step. I use a property of the \href{https://en.wikipedia.org/wiki/Gamma_function}{gamma function}. Remember, if $W \sim \bernoulli{\theta}$ then $\prob{W=1} = \theta$. Use that trick! Set $X^* = 1$ and find that probability!}\spc{12}


\easysubproblem{Assume the dataset in (b) where $n=6$. Assume $\theta \sim \betanot{\alpha=2}{\beta=2}$ a priori. Find the $\thetahatmap$, $\thetahatmmse$ and $\thetahatmae$ estimates for $\theta$. For the $\thetahatmae$ estimate, you'll need to obtain a quantile of the beta distribution. Use \texttt{R} on your computer or online using \href{http://rextester.com/l/r_online_compiler}{rextester}. The \texttt{qbeta} function in \texttt{R} finds arbitrary beta quantiles. Its first argument is the quantile desired e.g. 2.5\%, the next is $\alpha$ and the third is $\beta$. So to find the 97.5\%ile of a $\betanot{\alpha=2}{\beta=2}$ for example you type \texttt{qbeta(.975, 2, 2)} into the \texttt{R} console.}\spc{2}

\intermediatesubproblem{Why are all three of these estimates the same?}\spc{5}

\easysubproblem{Write out an expression for the 95\% credible region for $\theta$. Then solve computationally using the \texttt{qbeta} function in \texttt{R}.}\spc{3}

\easysubproblem{Compute a 95\% frequentist CI for $\theta$.}\spc{5}

\hardsubproblem{Let $\mu : \reals \rightarrow \reals^+$ be the \href{https://en.wikipedia.org/wiki/Lebesgue_measure}{Lebesgue measure} which measures the length of a subset of $\reals$. Why is $\mu(\text{CR}) < \mu(\text{CI})$? That is, why is the Bayesian Confidence Interval tighter than the Frequentist Confidence Interval? Use your previous answers. }\spc{5}

\easysubproblem{Explain the disadvantages of the highest density region method for computing credible regions.}\spc{3}

\intermediatesubproblem{Design a prior where you believe $\expe{\theta} = 0.5$ and you feel as if your belief represents information contained in five coin flips.}\spc{3}

\intermediatesubproblem{Calculate a 95\% a priori credible region for $\theta$. Use \texttt{R} on your computer (or \href{https://rdrr.io/snippets/}{rdrr.io} online) and its \texttt{qbeta} function.}\spc{3}

\easysubproblem{You flip the same coin 100 times and you observe 39 heads. Calculate a 95\% a posteriori credible region for $\theta$. Round to the nearest 3 decimal points.}\spc{0.5}


\end{enumerate}


\end{document}