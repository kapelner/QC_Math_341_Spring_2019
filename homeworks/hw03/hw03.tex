\documentclass[12pt]{article}

\include{preamble}

\newtoggle{professormode}
\toggletrue{professormode} %STUDENTS: DELETE or COMMENT this line



\title{MATH 341 / 650.3 Spring 2019 Homework \#3}

\author{Professor Adam Kapelner} %STUDENTS: write your name here

\iftoggle{professormode}{
\date{Due in KY604, Wednesday 11:59PM, March 6, 2019 \\ \vspace{0.5cm} \small (this document last updated \today ~at \currenttime)}
}

\renewcommand{\abstractname}{Instructions and Philosophy}

\begin{document}
\maketitle

\iftoggle{professormode}{
\begin{abstract}
The path to success in this class is to do many problems. Unlike other courses, exclusively doing reading(s) will not help. Coming to lecture is akin to watching workout videos; thinking about and solving problems on your own is the actual ``working out.''  Feel free to \qu{work out} with others; \textbf{I want you to work on this in groups.}

Reading is still \textit{required}. For this homework set, read about the beta-binomial model, conjugacy, credible regions, hypothesis tests and the posterior predictive distribution.

The problems below are color coded: \ingreen{green} problems are considered \textit{easy} and marked \qu{[easy]}; \inorange{yellow} problems are considered \textit{intermediate} and marked \qu{[harder]}, \inred{red} problems are considered \textit{difficult} and marked \qu{[difficult]} and \inpurple{purple} problems are extra credit. The \textit{easy} problems are intended to be ``giveaways'' if you went to class. Do as much as you can of the others; I expect you to at least attempt the \textit{difficult} problems. 

Problems marked \qu{[MA]} are for the masters students only (those enrolled in the 650.3 course). For those in 341, doing these questions will count as extra credit.

This homework is worth 100 points but the point distribution will not be determined until after the due date. See syllabus for the policy on late homework.

Up to 10 points are given as a bonus if the homework is typed using \LaTeX. Links to instaling \LaTeX~and program for compiling \LaTeX~is found on the syllabus. You are encouraged to use \url{overleaf.com}. If you are handing in homework this way, read the comments in the code; there are two lines to comment out and you should replace my name with yours and write your section. The easiest way to use overleaf is to copy the raw text from hwxx.tex and preamble.tex into two new overleaf tex files with the same name. If you are asked to make drawings, you can take a picture of your handwritten drawing and insert them as figures or leave space using the \qu{$\backslash$vspace} command and draw them in after printing or attach them stapled.

The document is available with spaces for you to write your answers. If not using \LaTeX, print this document and write in your answers. I do not accept homeworks which are \textit{not} on this printout. Keep this first page printed for your records.

\end{abstract}

\thispagestyle{empty}
\vspace{1cm}
NAME: \line(1,0){380}
\clearpage
}

\problem{Assume $\mathcal{F} =$ binomial with $n$ fixed and $\prob{\theta} = \betanot{\alpha}{\beta}$. To solve numerically, use \texttt{R} on your computer (or \href{https://rdrr.io/snippets/}{rdrr.io} online).}

\begin{enumerate}
\intermediatesubproblem{Design a prior where you believe $\expe{\theta} = 0.5$ and you feel as if your belief represents information contained in five coin flips.}\spc{4}

\intermediatesubproblem{You flip the same coin 100 times and you observe 39 heads. Test the hypothesis that this coin is fair given prior information from (a). Write out the hypotheses for this test, declare an $\alpha$ and use the credible region method. Make sure you say whether you retain or reject the null and justify why.}\spc{12}


\intermediatesubproblem{Let's say you wanted to test whether the coin is fair but you are indifferent to any $\theta$ which is different from fair by a margin of $\delta$ that you pick. Write out the hypotheses for this test, declare an $\alpha$ and calculate the Bayesian $p$-val for the test to determine if you should retain or reject $H_0$.}\spc{9}

\intermediatesubproblem{Test the hypothesis that this coin has a bias towards tails given prior information from (a) and the data from (b). Write out the hypotheses for this test, declare an $\alpha$ and calculate the Bayesian $p$-val for the test to determine if you should retain or reject $H_0$.}\spc{9}

\easysubproblem{Assume again the prior information from (a). What is the shrinkage proportion $\rho$ for this prior when estimating $\theta$ via $\thetahatmmse$?.}\spc{2}

\hardsubproblem{Prove that $\thetahatmmse$ is a biased estimator (i.e. its expectation is \textit{not} $\theta$).}\spc{4}

\easysubproblem{Prove that $\displaystyle \limitn \rho = 0$ and therefore this bias $\rightarrow 0$ as your dataset gets larger.}\spc{3}

\hardsubproblem{[MA] Why should anyone use shrinkage estimators if they're biased? Google it. Discuss.}\spc{10}

\hardsubproblem{Find the posterior predictive distribution, $X^*~|~X$. MA students --- do this yourself. Other students --- use my notes and justify each step. Remember, if $W \sim \bernoulli{p}$ then $\prob{W=1} = p$. Use that trick.}\spc{10}

\hardsubproblem{Using the prior in (a), can $X^*~|~X$ ever be degenerate? Yes/no and explain.}\spc{3}


\hardsubproblem{Using the general prior $\prob{\theta} = \betanot{\alpha}{\beta}$, can $X^*~|~X$ ever be degenerate? Yes/no and explain.}\spc{2.5}

\intermediatesubproblem{Given the prior information in (a) and the data in (b), what is the probability the next flip is a tail?}\spc{0}


\end{enumerate}


\end{document}

